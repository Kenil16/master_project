\documentclass[../Head/Report.tex]{subfiles}
\begin{document}
\section{Introduction}


When drones are set to fly autonomously they rely heavily on the Global Navigation System (GPS). Most of these systems comes with an error in the range of meters. To achieve better, real time kinematics (RTK) can be used, which reduces the GPS error to centimeters. However, RTK is pretty expensive and hence not an optimal solution for low cost applications. Moreover, for indoor navigation, the use of GPS would not be possible because of the  reduction in signal strength. 

A solution to this problem could be to use a camera placed on the drone and analyzing the incoming data using computer vision. By placing markers on the ground, the position of these objects could be found with high precision. This would lead to autonomous flight tasks where high precision of the position is needed.      

This paper proposes methods for navigation in environments using computer vision. The basic idea of this can be seen in Figure \ref{fig:masterProjectIllustration}. Here the drone will fly autonomously using GPS coordinates until a higher accuracy is needed. In this case, it will be the navigation and precision landing of the drone using markers on the ground. Hence, the drone will follow the markers till a landing is required which is illustrated as step 3. The markers considered in this case will be Aruco markers, which have been shown to be very accurate and reliable in changing light conditions \cite{visualmarkers}. In this example, the same marker is used for illustration, but different bit encoded Aruco markers will be used to navigate the drone to the landing sight. This is a cheap and effective solution when lack of GPS precision is present e.g using low cost GPS systems or inside buildings (Hanger) \cite{Visual-Inertial-Navigation}. 

\begin{figure}[H]
	\centering
	\includegraphics[height=5cm]{../Figures/masterProjectIllustration.png}
	\captionsetup{justification=centering}
    \caption{Illustration of the steps of a drone navigated autonomously using GPS to indoor navigation using computer vision leading to a precision landing }
    \label{fig:masterProjectIllustration}
\end{figure}

The robot operating system (ROS) will be used in order to achieve autonomous flight of the drone with Gazebo as the simulation environment so that the drone can be sufficiently tested before flying. Testing of the drone will take place at Hans Christian Andersen Airport in Odense which is why the building labeled \textit{Hangar} is used in Figure \ref{fig:masterProjectIllustration}.   


The reason for choosing this area of subject is because of the ongoing HealthDrone project. This project is about transportat-ion of patient samples between Odense University Hospital (OUH) and Svenborg Hospital using drones autonomously. This is a three-year innovation project funded by Innovation Fund Denmark and is to be completed in 2021 \cite{HealthDrone}. 

To accomplish this, an efficient and robust solution for indoor navigation and landing must be considered for where battery recharging or replacement are to be performed as well as delivery of patient samples. The reduce delivery time, the indoor navigation and landing of the drone must be a fast as possible. Furthermore, because the drone is to land on a recharging or battery replacement station, the landing must be performed with a very high precision. Due to the fact that the drone is to be operated in changing weather conditions, it must be able to detect and decrease its altitude according to the markers on the ground in windy conditions. This demands for a robust solution for the transition of using GPS coordinates to computer vision for indoor navigation.



In regard to the hardware used in this project, a quad-copter with a DJI SK500 frame and Pixhawk 2.1 are being considered.   



\subsection{Problem Statement}

The drone must be able to fly autonomously according to markers on the ground using computer vision. The  This should lead to a high precision landing where the navigation is based on the position of the markers. 

This leads to the following problems:

\begin{itemize}
    \item How can computer vision be used to detect objects?
    \item How can navigation between objects be performed?
    \item How can a precise landing be executed?
    \item How can the landing be performed in windy conditions?
    
\end{itemize} 

\subsection{Specification of requirements}

From the outline of the project as well as the problem statement, the following requirements for the project have been formulated:

\begin{itemize}
    \item The error of the landing must not exceed $\pm 10$ cm  
    \item The drone must be able to land in windy conditions e.g up to 8 m/s
    \item The landing must be performed within 5 seconds from a flying height of 2 meters 

\end{itemize}


\end{document}